\subsection{Front-end}
\subsubsection{Intoduction}
\begin{flushleft}
Dans cette partie, nous allons vous expliquer la façon dont nous avons implémenté le front-end.
\end{flushleft}
\begin{flushleft}
Le fichier "index.js" permet d'identifier plus clairement les parties, nous avons travaillé en prenant en compte ces différences : 
\end{flushleft}
\begin{enumerate}[-]
\item \textbf{Pages liées à la gestion des comptes}
\item \textbf{Pages liées aux clients}
\item \textbf{Pages liées aux fournisseurs}
\item \textbf{Pages liées aux fonctionnalités communes}
\end{enumerate} 
\subsubsection{Pages liées à la gestion des comptes}
\begin{enumerate}
\item \textbf{Login} :\newline
Cette page permet aux utilisateurs de se connecter.
\item \textbf{CreateAccount} :\newline
Cette page permet aux utilisateurs de créer un compte.
\item \textbf{ForgottenPassword} :\newline
Cette page donne la possibilité aux utilisateurs de changer de mot de passe s'ils le souhaitent ou s'ils ne se souviennent plus de celui-ci.
\end{enumerate}

\subsubsection{Pages liées aux clients}
\begin{enumerate}
\item \textbf{HomeClient} :\newline
La page d’accueil des clients.
\item \textbf{Wallets} :\newline
Cette page comprend tous les portefeuilles du client.
\item \textbf{WalletFull} :\newline
Cette page permet de voir les informations liées à un portefeuille en particulier en partant de la page Wallets.
\item \textbf{AddWallet} :\newline
Cette page donne la possibilité au client d’ajouter un portefeuille.
\item \textbf{NewContracts} : \newline
Cette page permet au client d’envoyer une notification au fournisseur afin d’essayer d’avoir un nouveau contrat avec ce fournisseur en question.
\item \textbf{ContractInformation} : \newline
Cette page permet de voir les informations liées à une nouvelle proposition se trouvant dans NewContractsPage.
\item \textbf{ContractPage} : \newline
Cette page permet de voir les contrats en cours qu’un client a en commun avec un fournisseur.
\end{enumerate} 

\subsubsection{Pages liées aux fournisseurs}
\begin{enumerate}
\item \textbf{HomeSupplier} : \newline
La page d’accueil des fournisseurs.
\item \textbf{Clients} : \newline
Cette page comprend tous les clients du fournisseur.
\item \textbf{ClientFull} :  \newline
Cette page permet de voir les informations liées à un client en particulier en partant de la page Clients.
\item \textbf{AddClient} : \newline
Cette page donne la possibilité au fournisseur d’ajouter un client avec la proposition qu’il désire.
\item \textbf{CreateContract} :\newline
Cette page permet au fournisseur de créer des nouvelles propositions. \newline
Notez que si on entre "baba" en durée, une fois le contrat associé à un client, celui-ci durera une heure. 
\item \textbf{SupplierContractPage} :\newline
Cette page permet au fournisseur de voir les propositions qu’il a créées.
\item \textbf{ProposalFull} :\newline
Cette page permet de voir les informations liées à une proposition en particulier en partant de la page SupplierContractPage.
\item \textbf{SupplierModifyContract} :\newline
Cette page donne la possibilité au fournisseur de modifier ses propositions ou ses contrats en cours s’ils sont variables.
\end{enumerate} 

\subsubsection{Pages liées aux fonctionnalités communes}
\begin{enumerate}
\item \textbf{Notifications} :\newline
Cette page donne la possibilité aux utilisateurs de voir leurs notifications ainsi que les accepter ou les refuser.
\item \textbf{ContractFull} :\newline
Cette page permet aux utilisateurs de voir un contrat en cours de manière détaillée en partant de la page ClientFull ou WalletFull.
\item \textbf{ConsumptionPage} :\newline
Cette page permet aux utilisateurs d’observer la consommation liée à un contrat.
\end{enumerate}

\newpage
\subsection{Modifications apportées}
\begin{flushleft}
    Quelques modifications par rapport à la maquette fournie ont été apportées. En effet, lors de la conception de la maquette, nous ne connaissions pas les banques CSS mises à notre disposition et donc nous n'avions pas beaucoup pensé au design\footnote{Toutes les banques de CSS utilisées ont été citées dans le readme.}.
\end{flushleft}
\begin{flushleft}
    Dans l'ensemble, le design du site respecte tout de même la thématique que nous avions apporté lors de la réalisation de la maquette. Cependant quelques modifications de design comme : Les notifications, l'affichage des wallets/contrats/proposal, la suppression d'une barre de recherche, les formulaires et la page de login, ... Toutes ces choses ont reçu des modifications au niveau du design.
\end{flushleft} 

\subsubsection{Problèmes rencontrés}
\begin{flushleft}
Nous avons rencontré un souci avec notre router, l’ensemble des redirections fonctionnait en local mais pas sur alwaysdata en ligne.
\end{flushleft}
\begin{flushleft}
Nous n’avions pas compris que nous envoyions des fichiers statiques (venant du dist) sur le serveur.
\end{flushleft}
\begin{flushleft}
De ce fait, la configuration se faisait en nodejs mais pas en fichier statique, ce qui était la source du problème.
\end{flushleft}

\newpage
\subsubsection{Les langues}
\begin{flushleft}
    Pour respecter les normes d'internationalisation, nous avons du implémenter le module \textit{i18n}. Ce module nous permettait de traduire l'ensemble de notre site dans des langues que nous aurions prédéfinies. Ici, les langues ajoutées sont le français et l'anglais. C'est pourquoi, il n'y a aucun texte codé en "dur" mais nous faisons l'usage de \textit{clés de traduction} pour rediriger le site vers la bonne traduction en fonction de la langue stockée en base de données. La langue est chargée en base de donnée et est stockée localement dans la variable \textit{i18n.locale}.
\end{flushleft}
\begin{flushleft}
    Pour ajouter une langue, il suffit d'ajouter un fichier \textit{.json} dans avec les langues, de rajouter cette langue (via un ajout de ligne) dans le fichier json des langues configurées et ajouter les différentes clés de traduction.
\end{flushleft}

\subsubsection{Modules importés}
\begin{enumerate}[-]
\item \textbf{VueCookies} :\newline
Ce module est importé afin de gérer les cookies.
\item \textbf{VueSweetalert2} :\newline
Ce dernier est importé afin d'obtenir diverses pop-ups.
\item \textbf{i18n} :\newline
Celui-ci est importé dans le but de gérer les langues comme expliqué dans la section précédente.
\item \textbf{bluebird} :\newline
Ce module est importé pour permettre de laisser un temps d'attente.\newline
Celui-ci est utile notamment lorsqu'on souhaite afficher une pop-up et ensuite rediriger.
\end{enumerate} 
