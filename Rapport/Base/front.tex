\subsection{Front-end}
\subsubsection{Intoduction}
\begin{flushleft}
Dans cette partie, nous allons vous expliquer la façon dont nous avons implémenté le front-end.
\end{flushleft}
\begin{flushleft}
Le fichier "index.js" permet d'identifier plus clairement les parties, nous avons travaillé en prenant en compte ces différences : 
\end{flushleft}
\begin{enumerate}[-]
\item \textbf{Pages liées à la gestion des comptes}
\item \textbf{Pages liées aux clients}
\item \textbf{Pages liées aux fournisseurs}
\item \textbf{Pages liées aux fonctionnalités communes}
\end{enumerate} 
\newpage
\subsubsection{Pages liées à la gestion des comptes}

\subsubsection{Pages liées aux clients}

\subsubsection{Pages liées aux fournisseurs}

\subsubsection{Pages liées aux fonctionnalités communes}

\subsubsection{Problèmes rencontrés}
On est passé de vue2 à 3.
Router

\subsubsection{Modules importés}
