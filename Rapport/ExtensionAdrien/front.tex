\subsection{Front-end}

\begin{flushleft}
Au niveau frontend, il y a princalement 2 gros changements que j'ai dû effectuer.
Notez que j'ai également dû rajouter des traductions pour mon extension dans les fichiers fr.json et en.json.
\end{flushleft}

\begin{enumerate}
    \item Les informations du portefeuilles
    \item La page pour voir les consommations
\end{enumerate}

\subsubsection{Les portefeuilles}
\begin{flushleft}
Pour cette partie ci, j'ai rajouté des inputs lors de la création des portfeuilles afin d'y mettre plus d'information. On y retrouve:
\end{flushleft}

\begin{enumerate}
    \item Le nombre d'habitant
    \item La taille de l'habitation
    \item Si c'est une maison ou bien un appartement
    \item Si l'habitation est chauffé au gaz ou à l'électricité
    \item S'il y a des panneaux solaires
\end{enumerate}

\begin{flushleft}
J'ai donc également affiché toutes ces nouvelles informations lorsqu'on affiche toutes les données d'un portefeuille.
\end{flushleft}

\subsubsection{Consommations}
\begin{flushleft}
Cette seconde partie étant le centre de mon extension, c'est ici qu'il y a eu les plus gros changements. Afin de rescpecter les consignes de mon extension, j'ai rajouté à cette page un bouton à option en bas de page qui permet de choisir le mode d'affichage que l'on souhaite. On y retrouve:
\end{flushleft}

\begin{enumerate}
    \item (Just See): le mode de base
    \item (Compare with other): Pour comparer ses données avec quelqu'un d'autre (la simulation)
    \item (Compare over time): Pour pouvoir comparer ses données à deux endroits en même temps
    \item (Statistic): Pour voir les statistiques des consommations de l'utilisateur par rapport au données affiché actuellement sur le graphique
\end{enumerate}

\begin{flushleft}
Notez donc que j'ai finalement opté pour un seul nouveau bouton reprenant les différentes possibilités de mon extension, je n'ai pas gardé le fonctionnement emis lors de la partie modélisation.
\end{flushleft}

