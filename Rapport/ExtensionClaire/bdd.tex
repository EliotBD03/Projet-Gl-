\subsection{Base de données}

\subsubsection{Introduction}

\begin{flushleft}
Dans cette partie, je vais vous expliquer les modifications et ajouts apportés à la base de données de la base du projet. 
\end{flushleft}

\begin{flushleft}
La base de données se voit ajouter deux tables :
\end{flushleft}

\begin{enumerate}
\item \textbf{invitedTable} :\newline
Cette table permet de regrouper de manière claire les informations nécessaires à la gestion des clients invités.
\item \textbf{invitation} :\newline
Celle-ci permet de gérer les informations échangées pour les invitations aux portefeuilles.
\end{enumerate}

\begin{flushleft}
Cependant, il n'y a pas de modifications apportées aux tables de la base du projet.
\end{flushleft}

\newpage

\subsubsection{Package dataobject}

\begin{flushleft}
Vous pouvez observer l'ajout des classes :
\end{flushleft}
\begin{enumerate}[-]
\item \textbf{Invitation}
\item \textbf{InvitedClient}
\end{enumerate}

\begin{flushleft}
L'une permet de regrouper les informations des clients invités sur un portefeuille, l'autre permet de regrouper les informations sur les invitations envoyées d'un client à l'autre.
\end{flushleft}

\begin{flushleft}
Vous pouvez également remarquer l'ajout d'une liste de clients invités dans "WalletFull".
\end{flushleft}

\begin{flushleft}
En effet, c'est dans la page de "WalletFull" que la gestion des clients invités se fera plus tard.
\end{flushleft}

\begin{flushleft}
De plus, l'ajout de la possibilité d'avoir des permissions dans "WalletBasic" permet de séparer les invités pouvant lire ou lire et écrire des propriétaires.
\end{flushleft}

\begin{flushleft}
Effectivement, il suffit d'obtenir une fois les permissions côté front-end grâce au "WalletBasic" pour prendre connaissance de si la page "WalletFull" ou la page de consommations doit s'afficher de manière à répondre à un propriétaire, à un client invité en lecture ou à un client invité en lecture et écriture.
\end{flushleft}

\begin{flushleft}
Il n'est donc pas nécessaire de l'inclure à la classe "WalletFull".
\end{flushleft}
\newpage

\subsubsection{Package database}

\begin{flushleft}
Dans ce package, nous avons l'ajout de deux "Manager" ainsi qu'une légère modification de "WalletManager".
\end{flushleft}

\begin{enumerate}
\item \textbf{InvitationManager} :\newline
Ce "Manager" vient principalement du choix de séparer les notifications des invitations. \newline
Cette décision me permet de gérer de manière à part entière les ajouts dans mes deux tables notamment dans invitedTable.\newline
Cette manière de procéder permet aussi de garder une séparation entre les notifications (échangées entre les fournisseurs et les clients) et les invitations (échangées uniquement entre clients).\newline
La table "invitation" contient :
\begin{enumerate}[-]
\item \textbf{invitationId} :\newline
L'identifiant de l'invitation.
\item \textbf{senderId} :\newline
L'identifiant du client émetteur.
\item \textbf{receiverId} :\newline
L'identifiant du client receveur.
\item \textbf{address} :\newline
L'adresse du portefeuille concerné.
\item \textbf{permission} :\newline
La permission accordée.
\item \textbf{nameSender} :\newline
Le nom de l'émetteur.
\item \textbf{type} :\newline
Le type quant à lui permet de départager les invitations, les acceptations et les refus d'invitations.
Ce qui est utile côté front-end afin de déterminer s'il s'agit d'une invitation à accepter ou refuser ou d'un simple retour négatif ou positif à marquer comme lu.
\end{enumerate}

Vous pourrez donc retrouver dans cette classe les méthodes suivantes:
\begin{enumerate}
\item \textbf{createInvitation} :\newline
Cette méthode permet d'envoyer une invitation à un autre client ou simplement de répondre positivement ou négativement à cette dernière.\newline
Notez que cette méthode permet aussi de vérifier si le client a entré un identifiant existant et s'il n'a pas essayé de s'ajouter lui-même.
\item \textbf{refuseInvitation} :\newline
Cette méthode donne la possibilité de refuser une invitation en renvoyant à l'émetteur que sa demande a été refusée.
\item \textbf{acceptInvitation} :\newline
Cette méthode donne la possibilité d'accepter une invitation en renvoyant à l'émetteur que sa demande a été acceptée.\newline
De plus, on prend en compte le retour de "addInvited" de la classe "InvitedClientManager".
\item \textbf{getAllInvitation} :\newline
Cette méthode permet d'obtenir toutes les invitations d'un client.\newline
Récupérer toutes ces informations permet en front-end de mieux gérer les explications données à ce dernier.
\item \textbf{deleteInvitation} :\newline
Cette méthode permet de supprimer une invitation ou une réponse positive ou négative.
\end{enumerate}

\item \textbf{InvitedClientManager} :\newline
Ce "Manager" est utilisé pour les actions à effectuer sur la table \textbf{"invitedTable"} en ce qui concerne les clients invités sur un portefeuille.\newline
La table "invitedTable" contient :
\begin{enumerate}[-]
\item \textbf{address} :\newline
L'adresse du portefeuille.
\item \textbf{invitedId} :\newline
L'identifiant du client invité.
\item \textbf{ownerId} :\newline
L'identifiant du client propriétaire.
\item \textbf{permission} :\newline
La permission accordée au client invité.
\end{enumerate}

Vous pourrez donc retrouver dans cette classe les méthodes suivantes:
\begin{enumerate}
\item \textbf{getAllInvitedClients} :\newline
Cette méthode permet d'obtenir la liste de tous les clients invités sur un portefeuille.\newline
Elle est utilisée dans "WalletManager" pour le "getWallet".
\item \textbf{deleteInvitedClient} :\newline
Cette méthode donne la possibilité de supprimer un invité d'un portefeuille.
\item \textbf{addInvited} :\newline
Cette méthode donne la possibilité d'ajouter un invité sur un portefeuille.\newline
Notez que cette dernière permet aussi de vérifier si le client n'était pas déjà invité sur le portefeuille.
\item \textbf{changePermission} :\newline
Cette méthode permet de modifier la permission donnée à un client invité sur un portefeuille.

\end{enumerate}

\end{enumerate}

\begin{flushleft}
L'ajout d'une méthode nommée \textbf{"getAllInvitedWallets"} dans \textbf{WalletManager} permet de corréler les éléments cités plus tôt dans le package dataobject.
Il y a maintenant une méthode pour obtenir tous les portefeuilles sous forme de "WalletBasic" sans permission pour le propriétaire et une autre avec permission pour les invités.
\end{flushleft}

\begin{flushleft}
Ces permissions étant reprises à l'aide de la table \textbf{"invitedTable"}.
\end{flushleft}

\begin{flushleft}
En outre, pour supprimer un portefeuille, il faut maintenant que le client n'ait plus de contrats mais aussi plus d'invités.
\end{flushleft}