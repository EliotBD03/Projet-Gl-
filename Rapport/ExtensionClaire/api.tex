\subsection{API}

\subsubsection{Introduction}

\begin{flushleft}
Dans cette partie, je vais vous expliquer les ajouts apportés à l'API de la base du projet. 
\end{flushleft}
\begin{flushleft}
Notez que les ajouts se trouvent dans ClientApi de ce package étant donné que cette extension ne concerne que les clients.
\end{flushleft}

\subsubsection{ClientApi}

\begin{flushleft}
Vous pouvez remarquer l'ajout des méthodes suivantes afin de faire la passerelle entre le front-end et la base de données.
\end{flushleft}

\begin{enumerate}
\item \textbf{getAllInvitedWallets} :\newline
Cette méthode permet d'appeler la méthode de "WalletManager" expliquée précédemment dans le but d'obtenir les portefeuilles "invités" avec la permission liée. 
\item \textbf{deleteInvitedClient} :\newline
Cette méthode permet d'appeler la méthode de "InvitedClientManager" afin de supprimer un invité. 
\item \textbf{changePermission} :\newline
Cette méthode permet d'appeler la méthode de "InvitedClientManager" afin de changer la permission un invité. 
\item \textbf{getAllInvitation} :\newline
Cette méthode permet d'appeler la méthode de "InvitationManager" dans le but d'obtenir toutes les invitations et refus ou acceptations de ces dernières. 
\item \textbf{proposeInvitation} :\newline
Cette méthode permet d'appeler la méthode de "InvitationManager" afin d'envoyer une invitation à un autre client pour espérer l'ajouter à la liste des invités. 
Cette méthode renvoie un code 500 si les conditions expliquées précédemment ne sont pas remplies.
\item \textbf{acceptInvitation} :\newline
Cette méthode permet d'appeler la méthode de "InvitationManager" afin d'accepter une invitation. 
Cette méthode renvoie aussi un code 500 si les conditions expliquées précédemment ne sont pas remplies.
\item \textbf{refuseInvitation} :\newline
Cette méthode permet d'appeler la méthode de "InvitationManager" afin de refuser une invitation. 
\item \textbf{deleteInvitation} :\newline
Cette méthode permet d'appeler la méthode de "InvitationManager" afin de supprimer une invitation. 
Cette dernière est également utile pour le front-end pour les invitations à marquer comme lues.
\end{enumerate}
