\subsection{Choix technologiques}
\begin{flushleft}
Tout d'abord, le language du language s'est porté sur le \textbf{java}. En dépis du fait qu'apprendre le Kotlin aurait été quelque chose d'extrèmement instructif, il aurait été impossible de fournir une application android.  
\end{flushleft}
\begin{flushleft}
    Ensuite, pour ce qui est de l'envoi/réception de requête. Il a été choisi de prendre le framework \textbf{retrofit}. Ce framework possède de multiples avantages tels que : apprentissage facile, dé/sérialisation automatique des objets, "type-safe", supporte plusieurs clients HTTP et il est extensible et customisable. En somme, il est très puissant.
\end{flushleft}
 \begin{flushleft}
     De plus, un design pattern d'architecture a été pris pour rendre le code le plus compréhensible possible. Celui-ci est le \textbf{Model-View-ViewModel} (MVVM). Celui-ci sépare la partie graphique, donnée et logique de l'application. Ceci nous permet d'avoir un code qui peut être facilement extensible.
 \end{flushleft}
 \begin{flushleft}
     Pour générer l'icone de l'application, un site web a été utilisé : \url{https://romannurik.github.io/AndroidAssetStudio/icons-launcher.html#foreground.type=clipart&foreground.clipart=android&foreground.space.trim=1&foreground.space.pad=0.25&foreColor=rgba(96%2C%20125%2C%20139%2C%200)&backColor=rgb(68%2C%20138%2C%20255)&crop=0&backgroundShape=circle&effects=none&name=ic_launcher}
 \end{flushleft}
 \begin{flushleft}
     Bien évidemment, le format du code de l'application est adapté à une app android.
 \end{flushleft}