\subsection{Difficultés rencontrées}
\begin{flushleft}
La raison principale du peu de contenu produit a été évidemment le temps. En effet, l'extension a commencé à être produite une semaine avant la remise de celle-ci. Bien qu'il aurait été possible d'y consacrer la majeure partie de la journée, la santée et l'union des cours priment.
\end{flushleft}
\begin{flushleft}
Ensuite, le fait de programmer en android n'est pas du tout une tâche aisée. Même si le java était déjà acquis depuis bien longtemps, de nouveaux aspects tels que les layouts, framework android, retrofit... se sont introduits dans le developpement nécéssitant de consacrer son temps à énormément de documentation. De plus, l'arborescence d'un projet android est très disparate.
\end{flushleft}
\begin{flushleft}
Enfin, même si l'extension compte pour 40\% de la note. Il faut avoir une base solide prête à toute épreuve. C'est pourquoi il a été choisi de se concentrer plus sur la base que l'extension.
\end{flushleft}