\subsection{API}

\begin{flushleft}
Nous retrouvons ici toutes les fonctionnalités implémentées par rapport à l'API. Bien entendu, nous ne reviendrons pas sur la partie commune qui est la même qu'expliqué précédemment. Ceci n'est qu'un explication des ajouts fait à cette API.
\end{flushleft}

\begin{flushleft}
    Tout d'abord, dans la class \textit{ClientAPI} nous retrouvons ces différentes méthodes. Toutes ces méthodes répondent au même fonctionnement : Recevoir une requête via le paramètre \textit{routingContext} et en retirer l'id, le body et parfois un élément présent dans la requête comme un id client. Ensuite, ces méthodes appeleront les Managers \textit{InvoiceManager} pour les factures et \textit{BankManager} pour toute la gestion du compte bancaire.
\end{flushleft}

\begin{enumerate}
    \item getAllInvoices :\newline
    Permet de recevoir la liste des \textit{InvoiceBasic}\footnote{Voir section Base de Données} lié à un client.
    \item getInvoice :\newline
    Permet de recevoir une \textit{InvoiceFull} avec plus d'informations
    \item changePaymentMethod :\newline
    Permet de changer la méthode de paiement (Manuel ou Automatique)
    \item changeAccountInformation :\newline
    Permet de changer les informations bancaires.
    \item changeProposal :\newline
    Permet de changer la proposition d'accompte mensuel faite par le client.
    \item addBank :\newline
    Permet de créer une entrée bancaire en base de données lors de la création d'un client.
    \item getBank :\newline
    Permet de récupérer les informations bancaires d'un client.
    \item changePaid :\newline
    Permet d'effectuer le paiement d'une facture.
\end{enumerate}


